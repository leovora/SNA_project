\section{Problem and Motivation}
\label{problem-and-motivation}

% What are the problems you want to address? Why are those problems
% important (impact, theoretical and/or practical needs, etc.)? What are
% the main contributions of the project?

The main objective of this study is to analyze and compare the structural and social dynamics of two covert criminal networks, one Italian and one London-based, in order to understand how relational structure, along with ethnic and national composition, influences internal organization and group resilience. Historically, researchers often overlooked gangs as collective entities, focusing instead on the traits of individual members \cite{1-McGloinDecker2010}. As a result, the connection between gang activity and gang structure has remained largely unexplored, described as a “black box” \cite{2-DeckerKatzWebb2008}. Nevertheless, a small but expanding literature on gangs and social networks has emerged, based on the idea that “human relationships form the least common denominator for organized crime” \cite{3-McIllwain1999}. 
\\ \\ Klein and Maxson \cite{4-KleinMaxson2006} highlight that “ethnicity is one of the most widely discussed, and little studied, aspects of gangs.” Ethnic groups—defined by shared heritage, culture, language, religion, or country of birth \cite{5-Smith1987} — play a central role in gang dynamics. Research has consistently shown that ethnicity is a key factor in both gang formation and membership \cite{6-Adamson2000} \cite{7-Alonso2004} \cite{8-FrengEsbensen2007} . While ethnic diversity has been recognized as influential in broader community contexts \cite{9-PyroozFoxDecker2010}, its specific relationship to gang activity and internal organization remains largely unexplored \cite{10-StarbuckHowellLindquist2001}.

The project therefore addresses two key issues: first, we seek to understand the structural organization and internal dynamics of covert criminal networks, focusing on how individual positions and relational patterns contribute to the stability and hierarchy of such groups; second, we aim to investigate the sociocultural dimension of these networks, exploring how national composition (i.e., the nationality of members) influences patterns of connection, leadership, and cooperation within and across criminal organizations.
\\ \\ 



