\section{Problem and Motivation}
\label{problem-and-motivation}

% What are the problems you want to address? Why are those problems
% important (impact, theoretical and/or practical needs, etc.)? What are
% the main contributions of the project?

The main objective of this study is to analyze and compare the structural and social dynamics of two covert criminal networks, one Italian and one London-based, in order to understand how relational structure, along with ethnic and national composition, influences internal organization and group resilience. Historically, researchers often overlooked gangs as collective entities, focusing instead on the traits of individual members \cite{mcgloin2010theories}. As a result, the connection between gang activity and gang structure has remained largely unexplored, described as a “black box” \cite{decker2008understanding}. Nevertheless, a small but expanding literature on gangs and social networks has emerged, based on the idea that “human relationships form the least common denominator for organized crime” \cite{mcillwain1999organized}.
\\ \\ Klein and Maxson \cite{klein2006street} highlight that “ethnicity is one of the most widely discussed, and little studied, aspects of gangs.” Ethnic groups—defined by shared heritage, culture, language, religion, or country of birth \cite{smith1987ethnic} — play a central role in gang dynamics. Research has consistently shown that ethnicity is a key factor in both gang formation and membership \cite{adamson2017defensive} \cite{alonso2004racialized} \cite{freng2007race} . While ethnic diversity has been recognized as influential in broader community contexts \cite{pyrooz2010racial}, its specific relationship to gang activity and internal organization remains largely unexplored \cite{starbuck2001hybrid}.

The project therefore addresses two key issues: first, we seek to understand the structural organization and internal dynamics of covert criminal networks, focusing on how individual positions and relational patterns contribute to the stability and hierarchy of such groups; second, we aim to investigate the sociocultural dimension of these networks, exploring how national composition (i.e., the nationality of members) influences patterns of connection, leadership, and cooperation within and across criminal organizations.
\\ \\


To achieve these aims, the project adopts a Social Network Analysis (SNA) approach, which has a long-standing tradition in gang and organized crime research \cite{klein1967groups} \cite{short1965group}. Technological advancements have significantly expanded its potential \cite{fleisher2005fieldwork} \cite{morselli2009inside}. Reconstructing a gang as a social network involves linking each unit—whether a group or an individual—according to the type of relationship under investigation \cite{wasserman1994social}.
This dual approach is essential not only to advance the theoretical application of Social Network Analysis within criminology, but also to generate actionable insights for security policies and investigative strategies by identifying key nodes and structural vulnerabilities in criminal systems. Although prior research has examined individual network structures, comparative studies across distinct sociocultural contexts remain rare. From a practical perspective, understanding how relational configurations and cultural factors interact to influence organizational resilience can inform policymaking, intelligence analysis, and law enforcement interventions, facilitating the detection of central actors, weak points, and cohesive subgroups that sustain covert operations.
\\ \\
The main contributions of the project include:
\begin{enumerate}
    \item Comparative Perspective - We will conduct a systematic comparison between two real-world covert criminal networks (Italian and London-based), highlighting structural and social differences.
    \item Integration of Structural and Sociocultural Analysis - By combining network metrics with the nationality attribute, we aim to understand how social identity and homophily affect organizational patterns and leadership roles.
    \item 3.	Insights into Network Robustness - Through the evaluation of cohesion and vulnerability to the removal of central nodes, the project will provide a deeper understanding of how criminal organizations maintain resilience despite potential disruptions.
\end{enumerate}

This project aims to contribute to the understanding of relational and cultural dynamics that shape criminal networks, providing both a theoretical foundation and methodological tools for future research and practical applications in the field of security and crime prevention.