\section{Problem and Motivation}
\label{problem-and-motivation}

The study aims to compare the structural and social dynamics of two covert criminal networks, one Italian and one London-based, to understand how relational structure and ethnic/national composition influence internal organization and group resilience. Historically, researchers often overlooked gangs as collective entities, focusing instead on the traits of individual members \cite{mcgloin2010theories}. As a result, the connection between gang activity and gang structure has remained largely unexplored, described as a “black box” \cite{decker2008understanding}. Nevertheless, a small but expanding literature on gangs and social networks has emerged, based on the idea that “human relationships form the least common denominator for organized crime” \cite{mcillwain1999organized}.
\\ \\ Klein and Maxson \cite{klein2006street} highlight that “ethnicity is one of the most widely discussed, and little studied, aspects of gangs.” Ethnic groups—defined by shared heritage, culture, language, religion, or country of birth \cite{smith1987ethnic} — play a central role in gang dynamics. Research has consistently shown that ethnicity is a key factor in both gang formation and membership \cite{adamson2017defensive} \cite{alonso2004racialized} \cite{freng2007race} . While ethnic diversity has been recognized as influential in broader community contexts \cite{pyrooz2010racial}, its specific relationship to gang activity and internal organization remains largely unexplored \cite{starbuck2001hybrid}.

The project therefore addresses two key issues: 

\begin{description}
    \item[Structural organization and internal dynamics:] Analyze how individual positions and relational patterns affect the stability and hierarchy of covert criminal networks.
    \item[Sociocultural dimension:] Examine how national composition (members’ nationality) shapes connection patterns, leadership, and cooperation within and between criminal organizations.
\end{description}


To achieve these aims, the project adopts a Social Network Analysis (SNA) approach, which has a long-standing tradition in gang and organized crime research \cite{klein1967groups} \cite{short1965group}. Technological advancements have significantly expanded its potential \cite{fleisher2005fieldwork} \cite{morselli2009inside}. Reconstructing a gang as a social network involves linking each unit—whether a group or an individual—according to the type of relationship under investigation \cite{wasserman1994social}.
This dual approach advances the theoretical use of Social Network Analysis in criminology and provides practical insights for security policies and investigations by identifying key nodes and structural vulnerabilities. While prior research often focuses on individual networks, comparative studies across sociocultural contexts are rare. Understanding how relational structures and cultural factors shape organizational resilience can guide policymaking, intelligence analysis, and law enforcement in detecting central actors, weak points, and cohesive subgroups in covert operations.
\\ \\
The main contributions of the project include:
\begin{enumerate}
    \item Comparative Perspective - Systematic comparison of Italian and London-based covert criminal networks to highlight structural and social differences.
    \item Structural and Sociocultural Integration - Analysis of network metrics alongside nationality to understand the impact of social identity and homophily on organization and leadership.
    \item Network Robustness Insights - Assessment of cohesion and vulnerability to node removal to reveal how criminal organizations sustain resilience.
\end{enumerate}