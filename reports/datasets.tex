\section{Datasets}
\label{datasets}

% How did you gather the data? Did you digitise it? How? Is the material
% publicly available? What tools did you use 1) to handle (store,
% manipulate) the data and 2) to compute measures on the data?


Our analysis uses two publicly available datasets from the UCINET Software project’s repository of social networks: the \textbf{London Gang Network} (54\,nodes) and the \textbf{Italian Gangs Network} (67\,nodes). The London dataset was accessed from the UCINET “Covert Networks” collection \cite{londonGangSite}; the Italian dataset was obtained from the same repository \cite{italianGangsSite}. Both include person–person adjacency matrices and accompanying node-attribute tables, and are fully digitised, requiring no manual transcription.

\subsection{Digitisation and Data Handling}

For this study, each adjacency matrix was stored in CSV format (\texttt{LONDON\_GANG.csv} with a 54\,$\times$\,54 matrix; \texttt{ITALIAN\_GANGS.csv} with a 67\,$\times$\,67 matrix).

It is important to note that the original London dataset contained weighted edges representing the strength of ties: 1 (hang out together), 2 (co-offend together), 3 (co-offend together, serious crime), and 4 (co-offend together, serious crime, kin). However, to ensure structural comparability with the Italian network, which is an unweighted graph, these weights were removed. The London network was thus binarized, treating all values $\ge 1$ as a simple connection (1) and absence of ties as 0.

The London attributes file (\texttt{LONDON\_GANG\_ATTR.csv}) lists \texttt{Age}, \texttt{Birthplace}, \texttt{Residence}, \texttt{Arrests}, \texttt{Convictions}, \texttt{Prison}, \texttt{Music}, and \texttt{Ranking}; the Italian attributes file (\texttt{ITALIAN\_GANGS\_ATTR.csv}) includes \texttt{Nationality/Country of origin}. To enable cross-network comparisons, we used only a harmonised birthplace variable (London \texttt{Birthplace}; Italian \texttt{Nationality/Country of origin}). Data handling was performed in \textbf{Python} with \textbf{\texttt{pandas}}, using \texttt{pandas.read\_csv} to load matrices and attributes into \texttt{DataFrame}s.

\subsection{Computing Measures}

Network measures were computed with \textbf{\texttt{NetworkX}}. The adjacency \texttt{DataFrame}s were converted into graph objects via \texttt{nx.from\_pandas\_adjacency}, which then served as the basis for all subsequent analyses and visualisations.
