\section{Datasets}
\label{datasets}

% How did you gather the data? Did you digitise it? How? Is the material
% publicly available? What tools did you use 1) to handle (store,
% manipulate) the data and 2) to compute measures on the data?


Our analysis uses two publicly available datasets from the UCINET project: the
\textbf{London Gang Network} (54 nodes) and the \textbf{Italian Gangs Network}
(67 nodes) \cite{londonGangSite, italianGangsSite}. Both datasets provide
1--mode person–person adjacency matrices together with node attributes, and are
fully digitised.

The \textbf{London} dataset consists of a 54$\times$54 undirected valued matrix.
Nodes represent gang members, and edges encode the presence and strength of
criminal or social relations (e.g.\ co-offending, spending time together,
shared crew membership).

The \textbf{Italian} dataset is a 67$\times$67 undirected adjacency matrix.
Nodes correspond to individuals involved in mafia-related activities, while
edges represent generic relational ties within the organisation (collaboration,
interaction, or shared illicit activities).

In both cases, nodes represent individuals and edges represent the social or
criminal connections among them, forming the structural basis for our network
analysis.

\subsection{Digitisation and Data Handling}

For this study, each adjacency matrix was stored in CSV format (\texttt{LONDON\_GANG.csv} with a 54\,$\times$\,54 matrix; \texttt{ITALIAN\_GANGS.csv} with a 67\,$\times$\,67 matrix).

It is important to note that the original London dataset contained weighted edges representing the strength of ties: 1 (hang out together), 2 (co-offend together), 3 (co-offend together, serious crime), and 4 (co-offend together, serious crime, kin). However, to ensure structural comparability with the Italian network, which is an unweighted graph, these weights were removed. The London network was thus binarized, treating all values $\ge 1$ as a simple connection (1) and absence of ties as 0.

The London attributes file (\texttt{LONDON\_GANG\_ATTR.csv}) lists \texttt{Age}, \texttt{Birthplace}, \texttt{Residence}, \texttt{Arrests}, \texttt{Convictions}, \texttt{Prison}, \texttt{Music}, and \texttt{Ranking}; the Italian attributes file (\texttt{ITALIAN\_GANGS\_ATTR.csv}) includes \texttt{Nationality/Country of origin}. To ensure consistency between the two metadata structures and enable cross-network comparisons, we restricted our analysis to a single, conceptually aligned attribute capturing individuals' origin (London \texttt{Birthplace}; Italian \texttt{Nationality/Country of origin}). Data handling was performed in \textbf{Python} with \textbf{\texttt{pandas}}, using \texttt{pandas.read\_csv} to load matrices and attributes into \texttt{DataFrame}s.

\subsection{Computing Measures}

Network measures were computed with \textbf{\texttt{NetworkX}}. The adjacency \texttt{DataFrame}s were converted into graph objects via \texttt{nx.from\_pandas\_adjacency}, which then served as the basis for all subsequent analyses and visualisations.
