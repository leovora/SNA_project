\section{Conclusion}
\label{conclusion}

Our comparative investigation ultimately reveals two distinct models of covert criminal organizations which, however, share underlying structural patterns and properties.

First, concerning the networks’ topologies and their global structural and organizational properties, our analysis demonstrated that the Italian gang operates as a more decentralized, modular organization, characterized by a higher modularity and a steeper hierarchy. On the other hand, the London gang functions as a denser, more cohesive centralized core, with a lower modularity and a flatter leadership structure.

Regarding the influence of ethnicity on tie formation, it emerged from this study that both the analyzed networks demonstrated a fundamental functional reliance on cross-ethnic relationships, evidenced by low ethnic homophily and solid rates of cross-ethnic ties. This suggests that, although existing literature \cite{adamson2017defensive} \cite{alonso2004racialized} \cite{freng2007race} often highlights ethnicity as a major driver for tie formation, individual network instances—such as those examined here—may exhibit different patterns. This does not contradict broader findings, but rather illustrates that complex mechanisms like ethnic homophily may manifest unevenly across contexts. 

An important distinction lies in the community structure of the two networks: the Italian network maintains high ethnic diversity within the communities, but presents a single, dominant ethnic group, while the London network, despite its overall integration, internally clusters into more ethnically homogeneous communities.

In conclusion, this analysis highlights that, as regards the examined network instances, ethnicity is not a dominant force able to shape tie formation or dictate interactions, but rather interplays complexly with the other network characteristics to shape patterns of cooperation, influence and community organization.