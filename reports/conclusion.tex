\section{Conclusion}
\label{conclusion}

Our comparative investigation ultimately reveals two distinct models of covert criminal organizations which, however, share underlying structural patterns and properties.\\

First, concerning the networks’ topologies and their global structural and organizational properties, our analysis demonstrated that the Italian gang operates as a more decentralized, modular organization, characterized by a higher modularity and a steeper hierarchy. On the other hand, the London gang functions as a denser, more cohesive centralized core, with a lower modularity and a flatter leadership structure.\\

Second, it emerged from this study that, contrary to similar studies [] which deem ethnicity as a primary driver the formation of ties between gangs members, both networks demonstrated a fundamental functional reliance on cross-ethnic ties, evidenced by low ethnic homophily and solid rates of cross-ethnic ties.\\

An important distinction lies in the community structure of the two networks: the Italian network maintains high ethnic diversity within the communities, but presents a single, dominant ethnic group, while the London network, despite its overall integration, internally clusters into more ethnically homogeneous communities.\\

In conclusion, this analysis highlights that ethnicity is not a dominant force able to shape tie formation or dictate interactions in the analyzed networks, but rather interacts complexly with the other network characteristics to shape patterns of cooperation, influence and community organization.\\