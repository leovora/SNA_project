\section{Measures and Results}
\label{measures}

% What measures did you apply (brief explanation of how they work)? How do
% they relate to the intent of the study? Why are they relevant? What is the connection among the gathered data, the applied measures,
% and the properties found?\\

For both the Italian and London gangs, we study the same network metrics and compare them. We start by analyzing \textbf{general structural metrics}, which quantify the overall organization and efficiency of the network:

\subsection*{General Structural Metrics}

\begin{itemize}
    \item \textbf{Density ($D$)}: measures how many ties exist in the network compared to the maximum possible number.
          \begin{equation}
              D = \frac{2m}{n(n-1)}
          \end{equation}
          where $m$ is the number of existing edges and $n$ the number of nodes. High density indicates strong cohesion and low vulnerability to node removal.

    \item \textbf{Average degree ($\langle k \rangle$)}: the mean number of connections per node.
          \begin{equation}
              \langle k \rangle = \frac{1}{n}\sum_{i=1}^{n} k_i = \frac{2m}{n}
          \end{equation}
          It represents member activity and overall engagement.

    \item \textbf{Network diameter ($D_{max}$)} and \textbf{average path length ($\ell$)}: respectively, the maximum and the mean of the shortest path distances between all node pairs.
          \begin{equation}
              \ell = \frac{1}{n(n-1)} \sum_{i \neq j} d_{ij}
          \end{equation}
          They describe the efficiency of information or order transmission within the network.

    \item \textbf{Clustering coefficient ($C$)}: expresses the probability that two neighbors of a node are connected, revealing local cohesion or closed ``cells''.
          For node $i$:
          \begin{equation}
              C_i = \frac{2t_i}{k_i(k_i - 1)}
          \end{equation}
          where $t_i$ is the number of triangles passing through $i$. The overall clustering coefficient is the average of $C_i$ across all nodes:
          \begin{equation}
              C = \frac{1}{n} \sum_{i=1}^{n} C_i
          \end{equation}

    \item \textbf{Modularity ($Q$)}: quantifies the presence of well-defined communities:
          \begin{equation}\label{eq:modularity}
              Q = \frac{1}{2m}\sum_{ij}\left(A_{ij}-\frac{d_j d_i}{2m}\right)\delta(g_i, g_j)
          \end{equation}
          where $m$ is the total number of edges, $A_{ij}$ is the adjacency matrix, $d$ represents node degrees, and $\delta(g_i, g_j)$ is the Kronecker delta (1 if nodes $i$ and $j$ share the same birthplace group, 0 otherwise). High $Q$ indicates strong internal divisions or cliques.
\end{itemize}

\subsection*{Centrality Metrics}

We also compute \textbf{centrality metrics}, which identify key actors based on different notions of importance:

\begin{itemize}
    \item \textbf{Degree centrality ($C_D(i)$)}: number of direct ties a node has.
          \begin{equation}
              C_D(i) = k_i
          \end{equation}
          It reveals the most active or visible members.

    \item \textbf{Betweenness centrality ($C_B(i)$)}: measures how often a node lies on the shortest paths between others.
          \begin{equation}
              C_B(i) = \sum_{s \neq i \neq d} \frac{\sigma_{sd}(i)}{\sigma_{sd}}
          \end{equation}
          where $\sigma_{sd}$ is the number of shortest paths between $s$ and $d$, and $\sigma_{sd}(i)$ those passing through $i$. It highlights brokers or gatekeepers.

    \item \textbf{Closeness centrality ($C_C(i)$)}: reciprocal of the mean distance from a node to all others.
          \begin{equation}
              C_C(i) = \frac{n-1}{\sum_{j\neq i} d_{ij}}
          \end{equation}
          Nodes with high $C_C$ can quickly disseminate information.

    \item \textbf{Eigenvector centrality ($C_E(i)$)}: assigns importance based on being connected to other important nodes.
          \begin{equation}
              C_E(i) = \frac{1}{\kappa}\sum_{j=1}^{n} A_{ij} C_E(j)
          \end{equation}
          where $\kappa$ is the largest eigenvalue of the adjacency matrix $A$. It identifies leaders recognized by other influential members.
\end{itemize}

\subsection*{Roles and Network Hierarchy}

Based on these metrics, we classify network roles as follows:

\begin{itemize}
    \item \textbf{Leaders}: nodes in the top 5\% for both degree and eigenvector centrality, broadly connected and influential.
    \item \textbf{Brokers}: nodes in the top 5\% for betweenness, key intermediaries between subgroups.
    \item \textbf{Peripherals}: nodes in the bottom 5\% for degree, few connections and limited influence.
\end{itemize}

We further analyze the \textbf{$k$-core decomposition} to inspect hierarchical structure and power concentration.
A $k$-core is a maximal subgraph where each node has degree $\geq k$.
The \textit{main core} corresponds to the highest $k$ with non-empty core.

We also assess \textbf{network robustness} by comparing metrics (density, average path length, $k$-core) before and after removing central nodes, in order to study the structural impact of node removal.

\subsection*{Attribute-based Network Analysis}

Finally, to explore how ethnic background shapes internal organization, we perform \textbf{attribute-based network analyses} based on the \textit{Birthplace} attribute.

\begin{itemize}
    \item \textbf{Assortativity (Modularity $Q$) and mixing matrix}: used to quantify the extent of homophily by \textit{Birthplace}. For unordered characteristics, assortativity is measured as the modularity $Q$ [\ref{eq:modularity}] indicating the extent to which similar nodes connect to each other.

    \item \textbf{Centrality by group}: comparing average degree, betweenness, and eigenvector centrality across ethnic categories reveals whether certain groups occupy more central positions.

    \item \textbf{Community composition and diversity}: community detection via modularity maximization; ethnic heterogeneity measured with the \textbf{Shannon Diversity Index}:
          \begin{equation}
              H = - \sum_i p_i \log(p_i)
          \end{equation}
          where $p_i$ is the proportion of members from group $i$ within a community. Higher $H$ values indicate more diverse communities.

    \item \textbf{Inter-group connectivity}: proportion of edges linking individuals of different birthplace categories, indicating cross-ethnic integration.

    \item \textbf{Subgraph analysis by birthplace}: evaluation of intra-group cohesion through internal density and clustering coefficient.
\end{itemize}

\textit{Note:} Since the Italian network is disconnected, metrics such as diameter and average path length are computed on the largest connected component, including cases where central nodes are removed.