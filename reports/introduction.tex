\section{Introduction}
\label{introduction}

% The context includes: the general field (e.g., literature, history,
% archaeology, tourism, biology, forensics, religious studies); the
% specific application (e.g., literary analysis, quantitative history,
% genetics, virology, forensics intelligence, tourism planning, biblical
% quantitative studies).

The present study is situated within the broader field of social sciences, with a particular focus on the application of Social Network Analysis (SNA) to the investigation of covert criminal organizations. Social Network Analysis offers a methodological framework to explore how individuals within a group are connected, how influence and control are distributed, and how structural properties affect organizational resilience and vulnerability.
Our specific application concerns the comparative analysis of two covert criminal networks — one operating in Italy and the other in London — in order to explore both structural and sociocultural differences between them. The datasets used for this study, \textit{Italian Gangs Network} (67 nodes, 114 edges) (available online at \cite{12-UCINetItalianGangs}) and \textit{London Gang Network} (54 nodes, 315 edges) (available online at \cite{11-UCINetLondonGang}) describe internal relationships within each gang and share a common metadata attribute: the nationality of each individual.
