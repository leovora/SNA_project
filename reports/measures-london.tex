\subsection{London gang}

\subsubsection{Structural analysis}

\subsubsection{Ethnicity analysis}

The analysis of the London gang network shows a \textbf{weak but positive tendency toward ethnic homophily}, with an assortativity coefficient of 0.113. This indicates that individuals display a mild preference for forming ties with others sharing the same \textit{Birthplace}, although the overall network remains relatively integrated. \\

The \textbf{mixing matrix} confirms this observation: diagonal values (particularly for groups 1 and 3) are slightly higher, indicating intra-group cohesion, while off-diagonal entries remain substantial. This balance highlights the presence 
of numerous \textbf{cross-ethnic connections} within the gang’s structure. \\

\begin{table}[H]
\centering
\caption{Mixing matrix (proportion of connections between Birthplace groups) – London gang}
\label{tab:mixing_matrix_london}
\begin{tabular}{|c|c|c|c|c|}
\hline
 & 1 & 2 & 3 & 4 \\ \hline
1 & 0.111 & 0.060 & 0.076 & 0.041 \\ 
2 & 0.060 & 0.073 & 0.076 & 0.025 \\ 
3 & 0.076 & 0.076 & 0.146 & 0.043 \\ 
4 & 0.041 & 0.025 & 0.043 & 0.025 \\ 
\hline
\end{tabular}
\end{table}

When analyzing \textbf{centrality measures}, groups 1 and 4 emerge as the most central and 
structurally influential, with the highest mean degrees (0.286 and 0.267 respectively) and eigenvector centralities (0.152 and 0.135). Group 2 follows closely, while group 3 (despite being the largest) exhibits the lowest centrality values, suggesting a more peripheral or clustered role. This pattern implies that influence and connectivity are distributed across multiple ethnic groups, rather than concentrated in a single one. \\

\begin{figure}[H]
    \centering
    \includegraphics[width=1\linewidth]{images/london_centrality_by_birthplace.png}
    \label{fig:london_centrality}
\end{figure}

The \textbf{community analysis} identified four main communities, each with different degrees of 
ethnic diversity. Community 0 displays a relatively balanced composition (1: 38\%, 2: 10\%, 3: 29\%, 4: 24\%), while community 1 is dominated by groups 2 and 3. Communities 2 and 3 are less diverse, with community 2 almost entirely composed of group 3 members. The mean Shannon diversity index ($H = 0.886$) indicates moderate internal diversity, slightly lower than in the Italian network. \\

\begin{table}[H]
\centering
\caption{Community composition by Birthplace – London gang}
\label{tab:community_composition_london}
\begin{tabular}{|c|c|c|c|c|}
\hline
Community & 1 & 2 & 3 & 4 \\ \hline
0 & 0.38 & 0.10 & 0.29 & 0.24 \\ 
1 & 0.20 & 0.35 & 0.40 & 0.05 \\ 
2 & 0.00 & 0.14 & 0.86 & 0.00 \\ 
3 & 0.00 & 0.33 & 0.67 & 0.00 \\ 
\hline
\end{tabular}
\end{table}

Furthermore, 64.44\% of all connections occur between individuals of different 
\textit{Birthplace} categories, demonstrating a high degree of cross-ethnic integration. 
As in the Italian case, national origin does not appear to be a key organizing principle in the network’s structure. \\

Subgraph analysis by \textit{Birthplace} provides additional insight:
\begin{itemize}
    \item Groups \textbf{1} and \textbf{4} exhibit the highest internal density (0.530 and 0.533 respectively) 
    and clustering, indicating strong intra-group cohesion.
    \item Group \textbf{3}, while the largest (24 nodes), has a lower internal density (0.167), 
    suggesting looser internal connectivity and a more outward orientation.
    \item Group \textbf{2} shows intermediate density (0.348) and clustering, forming connections both 
    internally and across groups.
\end{itemize}

\begin{table}[H]
\centering
\caption{Subgraph-level statistics by Birthplace – London gang}
\label{tab:subgraph_london}
\begin{tabular}{|c|c|c|c|c|}
\hline
Birthplace & Nodes & Edges & Density & Clustering \\ \hline
1 & 12 & 35 & 0.530 & 0.658 \\ 
2 & 12 & 23 & 0.348 & 0.636 \\ 
3 & 24 & 46 & 0.167 & 0.506 \\ 
4 & 6 & 8 & 0.533 & 0.722 \\ 
\hline
\end{tabular}
\end{table}

\begin{figure}[H]
    \centering
    \includegraphics[width=1\linewidth]{images/london_subgraphs.png}
    \label{fig:london_subgraphs}
\end{figure}

In summary, the London gang network displays slightly lower diversity but comparable integration when compared to the Italian case. While certain groups form cohesive internal clusters, 
the overall structure is characterized by extensive cross-ethnic linkage and 
distributed influence across national backgrounds. 
Ethnicity, therefore, plays only a minor role in shaping the gang’s internal connectivity patterns.