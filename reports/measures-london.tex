\subsection{London Gang Analysis}
This section will report the results of the metrics introduced at the beginning of section \ref{measures}, with respect to the London gang network.
\subsubsection{Structural analysis}
In figure \ref{fig:london_net} a plot of the network is shown, where each node is colored according to its \textit{Birthplace} label.
\begin{figure} [H]
    \centering
    \includegraphics[width=1\linewidth]{images/london_network.png}
    \caption{London Network graph visualization. Each node is colored according to its birthplace label}
    \label{fig:london_net}
\end{figure}

\subsubsection*{Macro-level Cohesion and Structure}
% The ‘X’ column expands to fill the space and allows text wrapping.
\begin{tabularx}{\linewidth}{@{} l l X @{}}
    \toprule
    \textbf{Metric}        & \textbf{Result} & \textbf{Interpretation (What it means)}                                                       \\
    \midrule
    Density                & 0.2201          & The network is extremely dense and highly interconnected.                                     \\
    Average Degree         & 11.67           & On average, each member is connected to almost 12 others.                                     \\
    Average Path Length    & 2.05            & Any two members can reach each other in just 2 "hops" on average.                             \\
    Diameter               & 4               & The maximum separation between any two members is 4 "hops".                                   \\
    Avg. Clustering Coeff. & 0.6331          & The network is rich in tightly-knit local subgroups (cliques).                                \\
    Modularity             & 0.2665          & The network operates as a single, cohesive bloc; it is not fragmented into separate factions. \\
    \bottomrule
\end{tabularx}

\subsubsection*{Micro-level Centrality and Social Roles}
\paragraph{Leaders} The most influential, connected, and central members.
\begin{table}[H] % Contenitore tabella
    \centering
    \begin{tabular}{r r r r r}
        \toprule
        \textbf{Node} & \textbf{Degree} & \textbf{Betweenness} & \textbf{Closeness} & \textbf{Eigenvector} \\
        \midrule
        1             & 0.4717          & 0.1087               & 0.6543             & 0.2367               \\
        7             & 0.4717          & 0.0755               & 0.6543             & 0.2433               \\
        12            & 0.4717          & 0.0596               & 0.6386             & 0.2494               \\
        \bottomrule
    \end{tabular}
    \caption{Leader nodes (Degree $\ge$ 0.4594 \& Eigenvector $\ge$ 0.2357)} \label{tab:leaders}
\end{table}

\paragraph{Brokers} Members who act as "bridges" connecting different parts of the network.
\begin{table}[H] % Contenitore tabella
    \centering
    \begin{tabular}{r r r r r}
        \toprule
        \textbf{Node} & \textbf{Degree} & \textbf{Betweenness} & \textbf{Closeness} & \textbf{Eigenvector} \\
        \midrule
        1             & 0.4717          & 0.1087               & 0.6543             & 0.2367               \\
        7             & 0.4717          & 0.0755               & 0.6543             & 0.2433               \\
        4             & 0.3962          & 0.0725               & 0.6163             & 0.1747               \\
        \bottomrule
    \end{tabular}
    \caption{Broker nodes (Betweenness $\ge$ 0.0670)} \label{tab:brokers}
\end{table}

\paragraph{Peripheral Members} Marginal members with few connections, existing on the network's edges.
\begin{table}[H] % Contenitore tabella
    \centering
    \begin{tabular}{r r r r r}
        \toprule
        \textbf{Node} & \textbf{Degree} & \textbf{Betweenness} & \textbf{Closeness} & \textbf{Eigenvector} \\
        \midrule
        38            & 0.0377          & 0.0000               & 0.4109             & 0.0256               \\
        39            & 0.0377          & 0.0000               & 0.3926             & 0.0258               \\
        40            & 0.0377          & 0.0000               & 0.3557             & 0.0107               \\
        45            & 0.0377          & 0.0000               & 0.4015             & 0.0164               \\
        50            & 0.0377          & 0.0000               & 0.3557             & 0.0107               \\
        53            & 0.0377          & 0.0003               & 0.3681             & 0.0088               \\
        \bottomrule
    \end{tabular}
    \caption{Peripheral nodes (Degree $\le$ 0.0377)} \label{tab:peripheral}
\end{table}

\subsubsection*{Hierarchy and Vulnerability}
These measures test the power structure and resilience of the network.

\paragraph{K-Core Decomposition}
The most densely connected "core" of the network is identified.  The analysis reveals a main core with a \textbf{k-value of 11}. This core consists of \textbf{13 nodes} (out of 54). The nodes in this core are: \texttt{[1, 2, 7, 8, 9, 10, 11, 12, 21, 22, 23, 25, 29]}.
\begin{figure} [H]
    \centering
    \includegraphics[width=1\linewidth]{images/london_network_core.png}
    \caption{Core of London Network graph visualization.}
    \label{fig:london_core_net}
\end{figure}

\paragraph{Vulnerability Simulation}
We simulated the removal of the identified leaders (Nodes 1, 7, 12), representing a \textbf{5.56\% reduction in nodes}, and recalculated cohesion metrics. This measure connects roles (leaders) to cohesion (robustness). The results show the network is \textbf{extremely robust}: removing the top 3 leaders did not fragment the network (it remained connected). The network density was reduced to \textbf{0.1906}, and the average path length increased to \textbf{2.2180} (an \textbf{8.00\%} increase). The high density and clustering create redundancy, making the network resilient to targeted attacks.
\begin{figure} [H]
    \centering
    \includegraphics[width=1\linewidth]{images/london_network_vuln.png}
    \caption{London Network sub-graph visualization without leaders.}
    \label{fig:london_net_vuln}
\end{figure}

Overall, the London network turns out to be a \textbf{properties} "small-world" (path length 2.05), highly cohesive (density 0.22), locally clustered (clustering 0.63), but undivided (modularity 0.26) network. It possesses a clear hierarchy (k-core 11) and defined social roles (Leaders 1, 7, 12), all of which combine to make it exceptionally \textbf{robust}.

\subsubsection{Ethnicity and Community Analysis}

The London gang network exhibits a \textbf{weak but positive tendency toward ethnic homophily}, reflected by an assortativity coefficient of 0.113. Although individuals show a mild preference for connecting with same-\textit{Birthplace} peers, the network remains largely integrated.

The \textbf{mixing matrix} supports this interpretation: diagonal entries (especially for groups 1 and 3) are slightly higher, but substantial off-diagonal values reveal frequent \textbf{cross-ethnic connections}.

\begin{table}[H]
    \centering
    \label{tab:mixing_matrix_london}
    \begin{tabular}{|c|c|c|c|c|}
        \hline
          & 1     & 2     & 3     & 4     \\ \hline
        1 & 0.111 & 0.060 & 0.076 & 0.041 \\
        2 & 0.060 & 0.073 & 0.076 & 0.025 \\
        3 & 0.076 & 0.076 & 0.146 & 0.043 \\
        4 & 0.041 & 0.025 & 0.043 & 0.025 \\
        \hline
    \end{tabular}
    \caption{Mixing matrix (proportion of connections between Birthplace groups) – London gang}
\end{table}

Regarding \textbf{centrality}, groups 1 and 4 emerge as the most structurally prominent, showing the highest degree and eigenvector values (0.286/0.152 and 0.267/0.135). Group 2 follows with intermediate scores, while group 3—despite being the largest—displays the lowest centrality, suggesting a more peripheral or internally clustered role. Overall, influence is distributed across multiple ethnic groups rather than concentrated in one.

\begin{figure}[H]
    \centering
    \includegraphics[width=1\linewidth]{images/london_centrality_by_birthplace.png}
    \label{fig:london_centrality}
\end{figure}

The \textbf{community analysis} identifies four communities with varying degrees of ethnic diversity. Community 0 shows a balanced mix, while community 1 is dominated by ethnic groups 2 and 3. Communities 2 and 3 are less diverse, particularly community 2, almost entirely composed of group 3 individuals. The mean Shannon index ($H = 0.886$) reflects \textbf{moderate diversity} overall.

\begin{table}[H]
    \centering
    \label{tab:community_composition_london}
    \begin{tabular}{|c|c|c|c|c|}
        \hline
        Community & 1    & 2    & 3    & 4    \\ \hline
        0         & 0.38 & 0.10 & 0.29 & 0.24 \\
        1         & 0.20 & 0.35 & 0.40 & 0.05 \\
        2         & 0.00 & 0.14 & 0.86 & 0.00 \\
        3         & 0.00 & 0.33 & 0.67 & 0.00 \\
        \hline
    \end{tabular}
    \caption{Community composition by Birthplace – London gang}
\end{table}

A total of 64.44\% of edges link individuals from different \textit{Birthplace} groups, indicating \textbf{high cross-ethnic integration} and suggesting that national origin is not a primary organizing factor.

Subgraph analysis offers further insight:
\begin{itemize}
    \item Groups \textbf{1} and \textbf{4} show the strongest internal cohesion (densities 0.530 and 0.533).
    \item Group \textbf{3}, though the largest, has lower internal density (0.167), implying weaker internal connectivity.
    \item Group \textbf{2} displays intermediate cohesion, linking both internally and externally.
\end{itemize}

\begin{table}[H]
    \centering
    \label{tab:subgraph_london}
    \begin{tabular}{|c|c|c|c|c|}
        \hline
        Birthplace & Nodes & Edges & Density & Clustering \\ \hline
        1          & 12    & 35    & 0.530   & 0.658      \\
        2          & 12    & 23    & 0.348   & 0.636      \\
        3          & 24    & 46    & 0.167   & 0.506      \\
        4          & 6     & 8     & 0.533   & 0.722      \\
        \hline
    \end{tabular}
    \caption{Subgraph-level statistics by Birthplace – London gang}
\end{table}

\begin{figure}[H]
    \centering
    \includegraphics[width=1\linewidth]{images/london_subgraphs.png}
    \label{fig:london_subgraphs}
\end{figure}

In summary, the London gang network presents \textbf{moderate ethnic diversity} and \textbf{high levels of cross-ethnic integration}. Some Birthplace groups form cohesive internal clusters, but overall the structure remains highly interconnected across ethnic boundaries. Influence is distributed, and ethnicity plays only a limited role in shaping the network’s internal organization.
