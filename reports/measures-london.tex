\subsection{London gang}
\subsubsection{Structural analysis}
This section will report the results of the metrics mentioned in \ref{measures} regarding the London gang network 54 nodes, 315 edges.

In figure \ref{fig:london_net} a plot of the network is shown where each node is colored according to its birthplace label.

\begin{figure} [H]
    \centering
    \includegraphics[width=1\linewidth]{images/london_network.png}
    \caption{London Network graph visualization. Each node is colored according to its birthplace label}
    \label{fig:london_net}
\end{figure}

\subsubsection{Macro-level Cohesion and Structure}
These measures assess the overall "connectedness" and efficiency of the network as a whole.

% The ‘X’ column expands to fill the space and allows text wrapping.
\begin{tabularx}{\linewidth}{@{} l l X @{}}
    \toprule
    \textbf{Metric}        & \textbf{Result} & \textbf{Interpretation (What it means)}                                                       \\
    \midrule
    Density                & 0.2201          & The network is extremely dense and highly interconnected.                                     \\
    Average Degree         & 11.67           & On average, each member is connected to almost 12 others.                                     \\
    Average Path Length    & 2.05            & Any two members can reach each other in just 2 "hops" on average.                             \\
    Diameter               & 4               & The maximum separation between any two members is 4 "hops".                                   \\
    Avg. Clustering Coeff. & 0.6331          & The network is rich in tightly-knit local subgroups (cliques).                                \\
    Modularity             & 0.2665          & The network operates as a single, cohesive bloc; it is not fragmented into separate factions. \\
    \bottomrule
\end{tabularx}

\subsubsection{Micro-level Centrality and Social Roles}
These measures identify the most important nodes, allowing us to define social roles.

\paragraph{Leaders} The most influential, connected, and central members.
\begin{table}[H] % Contenitore tabella
    \centering
    \begin{tabular}{r r r r r}
        \toprule
        \textbf{Node} & \textbf{Degree} & \textbf{Betweenness} & \textbf{Closeness} & \textbf{Eigenvector} \\
        \midrule
        1             & 0.4717          & 0.1087               & 0.6543             & 0.2367               \\
        7             & 0.4717          & 0.0755               & 0.6543             & 0.2433               \\
        12            & 0.4717          & 0.0596               & 0.6386             & 0.2494               \\
        \bottomrule
    \end{tabular}
    \caption{Leader nodes (Degree $\ge$ 0.4594 \& Eigenvector $\ge$ 0.2357)} \label{tab:leaders}
\end{table}

\paragraph{Brokers} Members who act as "bridges" connecting different parts of the network.
\begin{table}[H] % Contenitore tabella
    \centering
    \begin{tabular}{r r r r r}
        \toprule
        \textbf{Node} & \textbf{Degree} & \textbf{Betweenness} & \textbf{Closeness} & \textbf{Eigenvector} \\
        \midrule
        1             & 0.4717          & 0.1087               & 0.6543             & 0.2367               \\
        7             & 0.4717          & 0.0755               & 0.6543             & 0.2433               \\
        4             & 0.3962          & 0.0725               & 0.6163             & 0.1747               \\
        \bottomrule
    \end{tabular}
    \caption{Broker nodes (Betweenness $\ge$ 0.0670)} \label{tab:brokers}
\end{table}

\paragraph{Peripheral Members} Marginal members with few connections, existing on the network's edges.
\begin{table}[H] % Contenitore tabella
    \centering
    \begin{tabular}{r r r r r}
        \toprule
        \textbf{Node} & \textbf{Degree} & \textbf{Betweenness} & \textbf{Closeness} & \textbf{Eigenvector} \\
        \midrule
        38            & 0.0377          & 0.0000               & 0.4109             & 0.0256               \\
        39            & 0.0377          & 0.0000               & 0.3926             & 0.0258               \\
        40            & 0.0377          & 0.0000               & 0.3557             & 0.0107               \\
        45            & 0.0377          & 0.0000               & 0.4015             & 0.0164               \\
        50            & 0.0377          & 0.0000               & 0.3557             & 0.0107               \\
        53            & 0.0377          & 0.0003               & 0.3681             & 0.0088               \\
        \bottomrule
    \end{tabular}
    \caption{Peripheral nodes (Degree $\le$ 0.0377)} \label{tab:peripheral}
\end{table}

\subsubsection{Hierarchy and Vulnerability}
These measures test the power structure and resilience of the network.

\paragraph{K-Core Decomposition}
The most densely connected "core" of the network is identified.  The analysis reveals a main core with a \textbf{k-value of 11}. This core consists of \textbf{13 nodes} (out of 54). The nodes in this core are: \texttt{[1, 2, 7, 8, 9, 10, 11, 12, 21, 22, 23, 25, 29]}.
\begin{figure} [H]
    \centering
    \includegraphics[width=1\linewidth]{images/london_network_core.png}
    \caption{Core of London Network graph visualization.}
    \label{fig:london_core_net}
\end{figure}

\paragraph{Vulnerability Simulation}
We simulated the removal of the identified leaders (Nodes 1, 7, 12), representing a \textbf{5.56\% reduction in nodes}, and recalculated cohesion metrics. This measure connects roles (leaders) to cohesion (robustness). The results show the network is \textbf{extremely robust}: removing the top 3 leaders did not fragment the network (it remained connected). The network density was reduced to \textbf{0.1906}, and the average path length increased to \textbf{2.2180} (an \textbf{8.00\%} increase). The high density and clustering create redundancy, making the network resilient to targeted attacks.
\begin{figure} [H]
    \centering
    \includegraphics[width=1\linewidth]{images/london_network_vuln.png}
    \caption{London Network sub-graph visualization without leaders.}
    \label{fig:london_net_vuln}
\end{figure}

\subsubsection{Synthesis: Connection Between Data, Measures, and Properties}

The \textbf{properties} are what we discovered: a "small-world" network (path length 2.05), highly cohesive (density 0.22), locally clustered (clustering 0.63), but undivided (modularity 0.26). It possesses a clear hierarchy (k-core 11) and defined social roles (Leaders 1, 7, 12), all of which combine to make it exceptionally \textbf{robust}.

\subsubsection{Ethnicity analysis}

The analysis of the London gang network shows a \textbf{weak but positive tendency toward ethnic homophily}, with an assortativity coefficient of 0.113. This indicates that individuals display a mild preference for forming ties with others sharing the same \textit{Birthplace}, although the overall network remains relatively integrated. \\

The \textbf{mixing matrix} confirms this observation: diagonal values (particularly for groups 1 and 3) are slightly higher, indicating intra-group cohesion, while off-diagonal entries remain substantial. This balance highlights the presence
of numerous \textbf{cross-ethnic connections} within the gang’s structure. \\

\begin{table}[H]
    \centering
    \caption{Mixing matrix (proportion of connections between Birthplace groups) – London gang}
    \label{tab:mixing_matrix_london}
    \begin{tabular}{|c|c|c|c|c|}
        \hline
          & 1     & 2     & 3     & 4     \\ \hline
        1 & 0.111 & 0.060 & 0.076 & 0.041 \\
        2 & 0.060 & 0.073 & 0.076 & 0.025 \\
        3 & 0.076 & 0.076 & 0.146 & 0.043 \\
        4 & 0.041 & 0.025 & 0.043 & 0.025 \\
        \hline
    \end{tabular}
\end{table}

When analyzing \textbf{centrality measures}, groups 1 and 4 emerge as the most central and
structurally influential, with the highest mean degrees (0.286 and 0.267 respectively) and eigenvector centralities (0.152 and 0.135). Group 2 follows closely, while group 3 (despite being the largest) exhibits the lowest centrality values, suggesting a more peripheral or clustered role. This pattern implies that influence and connectivity are distributed across multiple ethnic groups, rather than concentrated in a single one. \\

\begin{figure}[H]
    \centering
    \includegraphics[width=1\linewidth]{images/london_centrality_by_birthplace.png}
    \label{fig:london_centrality}
\end{figure}

The \textbf{community analysis} identified four main communities, each with different degrees of
ethnic diversity. Community 0 displays a relatively balanced composition (1: 38\%, 2: 10\%, 3: 29\%, 4: 24\%), while community 1 is dominated by groups 2 and 3. Communities 2 and 3 are less diverse, with community 2 almost entirely composed of group 3 members. The mean Shannon diversity index ($H = 0.886$) indicates moderate internal diversity, slightly lower than in the Italian network. \\

\begin{table}[H]
    \centering
    \caption{Community composition by Birthplace – London gang}
    \label{tab:community_composition_london}
    \begin{tabular}{|c|c|c|c|c|}
        \hline
        Community & 1    & 2    & 3    & 4    \\ \hline
        0         & 0.38 & 0.10 & 0.29 & 0.24 \\
        1         & 0.20 & 0.35 & 0.40 & 0.05 \\
        2         & 0.00 & 0.14 & 0.86 & 0.00 \\
        3         & 0.00 & 0.33 & 0.67 & 0.00 \\
        \hline
    \end{tabular}
\end{table}

Furthermore, 64.44\% of all connections occur between individuals of different
\textit{Birthplace} categories, demonstrating a high degree of cross-ethnic integration.
As in the Italian case, national origin does not appear to be a key organizing principle in the network’s structure. \\

Subgraph analysis by \textit{Birthplace} provides additional insight:
\begin{itemize}
    \item Groups \textbf{1} and \textbf{4} exhibit the highest internal density (0.530 and 0.533 respectively)
          and clustering, indicating strong intra-group cohesion.
    \item Group \textbf{3}, while the largest (24 nodes), has a lower internal density (0.167),
          suggesting looser internal connectivity and a more outward orientation.
    \item Group \textbf{2} shows intermediate density (0.348) and clustering, forming connections both
          internally and across groups.
\end{itemize}

\begin{table}[H]
    \centering
    \caption{Subgraph-level statistics by Birthplace – London gang}
    \label{tab:subgraph_london}
    \begin{tabular}{|c|c|c|c|c|}
        \hline
        Birthplace & Nodes & Edges & Density & Clustering \\ \hline
        1          & 12    & 35    & 0.530   & 0.658      \\
        2          & 12    & 23    & 0.348   & 0.636      \\
        3          & 24    & 46    & 0.167   & 0.506      \\
        4          & 6     & 8     & 0.533   & 0.722      \\
        \hline
    \end{tabular}
\end{table}

\begin{figure}[H]
    \centering
    \includegraphics[width=1\linewidth]{images/london_subgraphs.png}
    \label{fig:london_subgraphs}
\end{figure}

In summary, the London gang network displays slightly lower diversity but comparable integration when compared to the Italian case. While certain groups form cohesive internal clusters,
the overall structure is characterized by extensive cross-ethnic linkage and
distributed influence across national backgrounds.
Ethnicity, therefore, plays only a minor role in shaping the gang’s internal connectivity patterns.