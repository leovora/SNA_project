
\subsection{Italian gang}

\subsubsection{Structural analysis}

Using the Library Networkx \cite{networkx} from Python we studied different aspects of the italian gang network.

In figure \ref{fig:italian_net} a plot of the network is shown where each node is colored according to its country label.

\begin{figure} [H]
    \centering
    \includegraphics[width=0.7\linewidth]{images/italian_network.png}
    \caption{Italian Network graph visualization. Each node is colored according to its country label}
    \label{fig:italian_net}
\end{figure}
\subsubsection{Macro-level Cohesion and Structure}

As the italian network is not connected and diameter and average path length are not defined for not connected networks, we instead calculate these metrics for the largest connected component, which has 65 nodes (out of 67).

\begin{tabularx}{\linewidth}{@{} l l >{\raggedright\arraybackslash}X @{}}
    \toprule
    \textbf{Metric}        & \textbf{Result}                                                                               \\
    \midrule
    Density                & 0.0516                                                    \\
    Average Degree         & 3.0430                                                                                    \\
    Average Path Length (largest connected component)     & 3.012          \\
    Diameter (largest connected component)               & 6         \\
    Avg. Clustering Coeff. & 0.4347         \\
    Modularity             & 0.5561          \\
    \bottomrule
\end{tabularx}
\subsubsection{Micro-level Centrality and Social Roles}
These measures identify the most important nodes, allowing us to define social roles.

\paragraph{Leaders} The most influential, connected, and central members.

\begin{table}[H]
    \centering
    \begin{tabular}{r r r r r}
        \toprule
        \textbf{Node} & \textbf{Degree} & \textbf{Betweenness} & \textbf{Closeness} & \textbf{Eigenvector} \\
        \midrule
        19            & 0.3182          & 0.5558               & 0.5397             & 0.4394               \\
        63            & 0.2879          & 0.3633               & 0.4597             & 0.3110               \\
        18            & 0.2727          & 0.2881               & 0.4563             & 0.3784               \\
        \bottomrule
    \end{tabular}
    \caption{Leader nodes (Degree $\ge$ 0.1621 \& Eigenvector $\ge$ 0.2542)} \label{tab:leaders}
\end{table}


\paragraph{Brokers} Members who act as "bridges" connecting different parts of the network.
\begin{table}[H] % Contenitore tabella
    \centering
    \begin{tabular}{r r r r r}
        \toprule
        \textbf{Node} & \textbf{Degree} & \textbf{Betweenness} & \textbf{Closeness} & \textbf{Eigenvector} \\
        \midrule
        19            & 0.3182          & 0.5558               & 0.5397             & 0.4394               \\
        63            & 0.2879          & 0.3633               & 0.4597             & 0.3110               \\
        18            & 0.2727          & 0.2881               & 0.4563             & 0.3784               \\
        47            & 0.1515          & 0.1396               & 0.4280             & 0.2177               \\
        \bottomrule
    \end{tabular}
    \caption{Broker nodes (Betweenness $\ge$ 0.1369)} \label{tab:brokers}
\end{table}

\paragraph{Peripheral Members} Marginal members with few connections, existing on the network's edges.
\begin{table}[H]
    \centering
    \begin{tabular}{r r r r r}
        \toprule
        \textbf{Node} & \textbf{Degree} & \textbf{Betweenness} & \textbf{Closeness} & \textbf{Eigenvector} \\
        \midrule
        2             & 0.0152          & 0.0000               & 0.0152             & 0.0000               \\
        3             & 0.0152          & 0.0000               & 0.2443             & 0.0132               \\
        5             & 0.0152          & 0.0000               & 0.3119             & 0.0556               \\
        17            & 0.0152          & 0.0000               & 0.2463             & 0.0141               \\
        22            & 0.0152          & 0.0000               & 0.3487             & 0.0645               \\
        26            & 0.0152          & 0.0000               & 0.2941             & 0.0384               \\
        33            & 0.0152          & 0.0000               & 0.2473             & 0.0179               \\
        34            & 0.0152          & 0.0000               & 0.2619             & 0.0140               \\
        36            & 0.0152          & 0.0000               & 0.2941             & 0.0349               \\
        37            & 0.0152          & 0.0000               & 0.2473             & 0.0179               \\
        38            & 0.0152          & 0.0000               & 0.3487             & 0.0645               \\
        45            & 0.0152          & 0.0000               & 0.2443             & 0.0132               \\
        46            & 0.0152          & 0.0000               & 0.0152             & 0.0000               \\
        48            & 0.0152          & 0.0000               & 0.3134             & 0.0457               \\
        49            & 0.0152          & 0.0000               & 0.3119             & 0.0556               \\
        51            & 0.0152          & 0.0000               & 0.2434             & 0.0117               \\
        53            & 0.0152          & 0.0000               & 0.3487             & 0.0645               \\
        55            & 0.0152          & 0.0000               & 0.3134             & 0.0457               \\
        56            & 0.0152          & 0.0000               & 0.2941             & 0.0384               \\
        60            & 0.0152          & 0.0000               & 0.3487             & 0.0645               \\
        61            & 0.0152          & 0.0000               & 0.2941             & 0.0349               \\
        66            & 0.0152          & 0.0000               & 0.2324             & 0.0069               \\
        67            & 0.0152          & 0.0000               & 0.3119             & 0.0556               \\
        \bottomrule
    \end{tabular}
    \caption{Peripheral nodes (Degree $\le$ 0.0152)} \label{tab:peripheral}
\end{table}

\subsubsection{Hierarchy and Vulnerability}
These measures test the power structure and resilience of the network.

\paragraph{K-Core Decomposition}
The most densely connected "core" of the network is identified.  The analysis reveals a main core with a \textbf{k-value of 3}. This core consists of \textbf{20 nodes} (out of 63). The nodes in this core are: \texttt{[4, 8, 11, 12, 13, 15, 18, 19, 21, 24, 31, 32, 39, 41, 44, 47, 58, 59, 63, 64]}.
\begin{figure} [H]
    \centering
    \includegraphics[width=1\linewidth]{images/italian_network_core.png}
    \caption{Core of Italian Network graph visualization.}
    \label{fig:italian_core_net}
\end{figure}

\paragraph{Vulnerability Simulation}
Removing the three identified leaders (19, 63, 18) had a clear impact on the network structure. After their removal, the network broke into 18 components, and the largest connected part dropped from 65 to 24 nodes, showing a strong loss of cohesion. The density decreased from 0.0516 to \textbf{0.0288}, meaning fewer remaining connections overall. The average shortest path length increased from 3.0120 to \textbf{3.1341} (a \textbf{4.05\%} rise), indicating slightly longer communication chains among the surviving nodes.
Overall, the simulation shows that leader nodes play an important structural role, and removing them makes the network significantly more fragmented and less efficient.
\begin{figure} [H]
    \centering
    \includegraphics[width=1\linewidth]{images/italian_net_vuln.png}
    \caption{Italian Network sub-graph visualization without leaders.}
    \label{fig:italian_net_vuln}
\end{figure}

\subsubsection{Synthesis: Connection Between Data, Measures, and Properties}
The \textbf{Italian gang network} is sparse but moderately cohesive, with clear community divisions and a defined inner core (\textbf{k = 3}). Leadership is concentrated in a few nodes (19, 63, 18), supported by brokers who link subgroups. When these leaders are removed, the network becomes \textbf{fragmented}, revealing limited redundancy and a \textbf{moderate structural vulnerability} despite its local cohesion.

\subsubsection{Ethnicity analysis}

The analysis of the Italian gang network shows a \textbf{moderate tendency toward ethnic homophily}, with an assortativity coefficient of 0.150. This value suggests that individuals show a slight preference for connecting with others sharing the same \textit{Birthplace}, though overall the network remains relatively integrated. \\

The \textbf{mixing matrix} confirms this trend: while several diagonal values
(e.g., for groups 3, 4, and 5) are higher (indicating intra-group cohesion) many off-diagonal entries are also non-negligible. This demonstrates a considerable number of \textbf{cross-national connections}, supporting the idea of partial ethnic mixing.

\begin{table}[H]
    \centering
    \caption{Mean centrality by Birthplace – Italian gang}
    \label{tab:mixing_matrix_italy}
    \begin{tabular}{|c|c|c|c|c|c|c|c|c|c|}
        \hline
          & 1     & 2     & 3     & 4     & 5     & 6     & 7     & 8     & 9     \\ \hline
        1 & 0.018 & 0.000 & 0.031 & 0.004 & 0.004 & 0.000 & 0.022 & 0.009 & 0.000 \\
        2 & 0.000 & 0.000 & 0.000 & 0.000 & 0.000 & 0.000 & 0.000 & 0.000 & 0.004 \\
        3 & 0.031 & 0.000 & 0.096 & 0.053 & 0.031 & 0.000 & 0.061 & 0.004 & 0.004 \\
        4 & 0.004 & 0.000 & 0.053 & 0.228 & 0.013 & 0.009 & 0.035 & 0.000 & 0.004 \\
        5 & 0.004 & 0.000 & 0.031 & 0.013 & 0.009 & 0.000 & 0.026 & 0.000 & 0.009 \\
        6 & 0.000 & 0.000 & 0.000 & 0.009 & 0.000 & 0.000 & 0.000 & 0.000 & 0.000 \\
        7 & 0.022 & 0.000 & 0.061 & 0.035 & 0.026 & 0.000 & 0.000 & 0.000 & 0.000 \\
        8 & 0.009 & 0.000 & 0.004 & 0.000 & 0.000 & 0.000 & 0.000 & 0.000 & 0.000 \\
        9 & 0.000 & 0.004 & 0.004 & 0.004 & 0.009 & 0.000 & 0.000 & 0.000 & 0.000 \\
        \hline
    \end{tabular}
\end{table}

When analyzing \textbf{centrality measures}, group 5 clearly stands out as the most central and structurally influential, with the highest mean degree (0.159), betweenness (0.144), and eigenvector centrality (0.233). Groups 3 and 9 also exhibit moderate centrality levels, suggesting participation in brokerage or connective roles. In contrast, groups such as 2, 6, and 7 show minimal centrality values, occupying peripheral positions within the network.
Overall, influence appears somewhat concentrated but not monopolized by a single nationality. \\

\begin{figure}[H]
    \centering
    \includegraphics[width=1\linewidth]{images/italian_centrality_by_birthplace.png}
    \label{fig:placeholder}
\end{figure}

The \textbf{community analysis} identified five major communities.
Most of these display mixed ethnic compositions, with only one cluster (community 4) being entirely dominated by two groups (1 and 3). The mean Shannon diversity index ($H = 1.174$) indicates high internal heterogeneity, suggesting that communities are composed of members from multiple national origins rather than segregated along ethnic lines.\\

Furthermore, 64.91\% of all connections occur between individuals of different
Birthplace categories, a clear indicator of \textbf{strong cross-ethnic integration}.
This finding supports the interpretation that ethnic background is not a dominant organizing factor in the structure of the Italian gang. \\

Subgraph analysis by \textit{Birthplace} reveals additional information:
\begin{itemize}
    \item Group \textbf{4} (21 nodes, density = 0.124) forms the largest and most internally cohesive subcommunity.
    \item Group \textbf{5}, although small (2 nodes), is fully interconnected (density = 1.000),
          representing a tightly bonded dyad.
    \item Other groups (6, 7, 8, 9) show limited or no internal links, implying
          \textbf{dependence on inter-ethnic ties} for maintaining connectivity.
\end{itemize}

\begin{table}[H]
    \centering
    \caption{Subgraph-level statistics by Birthplace – Italian gang}
    \label{tab:subgraph_italy}
    \begin{tabular}{|c|c|c|c|c|}
        \hline
        \textbf{Country} & \textbf{Nodes} & \textbf{Edges} & \textbf{Density} & \textbf{Clustering} \\ \hline
        1                & 5              & 2              & 0.200            & 0.000               \\
        2                & 1              & 0              & 0.000            & 0.000               \\
        3                & 15             & 11             & 0.105            & 0.156               \\
        4                & 21             & 26             & 0.124            & 0.294               \\
        5                & 2              & 1              & 1.000            & 0.000               \\
        6                & 1              & 0              & 0.000            & 0.000               \\
        7                & 20             & 0              & 0.000            & 0.000               \\
        8                & 1              & 0              & 0.000            & 0.000               \\
        9                & 1              & 0              & 0.000            & 0.000               \\
        \hline
    \end{tabular}
\end{table}

\begin{figure}[H]
    \centering
    \includegraphics[width=1\linewidth]{images/italian_subgraphs.png}
    \label{fig:placeholder}
\end{figure}

In summary, the Italian gang exhibits \textbf{moderate homophily but high overall integration},
with collaboration patterns largely transcending national divisions.
Ethnicity appears to play a \textbf{secondary role} in shaping relational dynamics.