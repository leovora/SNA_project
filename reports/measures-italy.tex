
\subsection{Italian Gang Analysis}
This section will report the results of the metrics introduced at the beginning of section \ref{measures}, with respect to the Italian gang network.
\subsubsection{Structural analysis}

In figure \ref{fig:italian_net} a plot of the network is shown, where each node is colored according to its \textit{Country} label.

\begin{figure} [H]
    \centering
    \includegraphics[width=0.7\linewidth]{images/italian_network.png}
    \caption{Italian Network graph visualization. Each node is colored according to its country label}
    \label{fig:italian_net}
\end{figure}
\subsubsection*{Macro-level Cohesion and Structure}

The following table reports the values of the structural metrics for the Italian Network. Since the network is not connected and diameter and average path length are not defined for disconnected networks, we instead calculate these metrics for the largest connected component, which has 65 nodes (out of 67).\\

\begin{tabularx}{\linewidth}{@{} l l >{\raggedright\arraybackslash}X @{}}
    \toprule
    \textbf{Metric}        & \textbf{Result}                                                                               \\
    \midrule
    Density                & 0.0516                                                    \\
    Average Degree         & 3.0430                                                                                    \\
    Average Path Length (largest connected component)     & 3.012          \\
    Diameter (largest connected component)               & 6         \\
    Avg. Clustering Coeff. & 0.4347         \\
    Modularity             & 0.5561          \\
    \bottomrule
\end{tabularx}
\subsubsection*{Micro-level Centrality and Social Roles}
The following tables illustrate which nodes occupy the social roles of \textit{Leaders}, \textit{Brokers} and \textit{Peripheral Members}, based on their centrality metric values, which are also reported in the tables.

\paragraph{Leaders} The most influential, connected, and central members.

\begin{table}[H]
    \centering
    \begin{tabular}{r r r r r}
        \toprule
        \textbf{Node} & \textbf{Degree} & \textbf{Betweenness} & \textbf{Closeness} & \textbf{Eigenvector} \\
        \midrule
        19            & 0.3182          & 0.5558               & 0.5397             & 0.4394               \\
        63            & 0.2879          & 0.3633               & 0.4597             & 0.3110               \\
        18            & 0.2727          & 0.2881               & 0.4563             & 0.3784               \\
        \bottomrule
    \end{tabular}
    \caption{Leader nodes (Degree $\ge$ 0.1621 \& Eigenvector $\ge$ 0.2542)} \label{tab:leaders}
\end{table}


\paragraph{Brokers} Members who act as "bridges" connecting different parts of the network.
\begin{table}[H] % Contenitore tabella
    \centering
    \begin{tabular}{r r r r r}
        \toprule
        \textbf{Node} & \textbf{Degree} & \textbf{Betweenness} & \textbf{Closeness} & \textbf{Eigenvector} \\
        \midrule
        19            & 0.3182          & 0.5558               & 0.5397             & 0.4394               \\
        63            & 0.2879          & 0.3633               & 0.4597             & 0.3110               \\
        18            & 0.2727          & 0.2881               & 0.4563             & 0.3784               \\
        47            & 0.1515          & 0.1396               & 0.4280             & 0.2177               \\
        \bottomrule
    \end{tabular}
    \caption{Broker nodes (Betweenness $\ge$ 0.1369)} \label{tab:brokers}
\end{table}

\paragraph{Peripheral Members} Marginal members with few connections, existing on the network's edges.
\begin{table}[H]
    \centering
    \begin{tabular}{r r r r r}
        \toprule
        \textbf{Node} & \textbf{Degree} & \textbf{Betweenness} & \textbf{Closeness} & \textbf{Eigenvector} \\
        \midrule
        2             & 0.0152          & 0.0000               & 0.0152             & 0.0000               \\
        3             & 0.0152          & 0.0000               & 0.2443             & 0.0132               \\
        5             & 0.0152          & 0.0000               & 0.3119             & 0.0556               \\
        17            & 0.0152          & 0.0000               & 0.2463             & 0.0141               \\
        22            & 0.0152          & 0.0000               & 0.3487             & 0.0645               \\
        26            & 0.0152          & 0.0000               & 0.2941             & 0.0384               \\
        33            & 0.0152          & 0.0000               & 0.2473             & 0.0179               \\
        34            & 0.0152          & 0.0000               & 0.2619             & 0.0140               \\
        36            & 0.0152          & 0.0000               & 0.2941             & 0.0349               \\
        37            & 0.0152          & 0.0000               & 0.2473             & 0.0179               \\
        38            & 0.0152          & 0.0000               & 0.3487             & 0.0645               \\
        45            & 0.0152          & 0.0000               & 0.2443             & 0.0132               \\
        46            & 0.0152          & 0.0000               & 0.0152             & 0.0000               \\
        48            & 0.0152          & 0.0000               & 0.3134             & 0.0457               \\
        49            & 0.0152          & 0.0000               & 0.3119             & 0.0556               \\
        51            & 0.0152          & 0.0000               & 0.2434             & 0.0117               \\
        53            & 0.0152          & 0.0000               & 0.3487             & 0.0645               \\
        55            & 0.0152          & 0.0000               & 0.3134             & 0.0457               \\
        56            & 0.0152          & 0.0000               & 0.2941             & 0.0384               \\
        60            & 0.0152          & 0.0000               & 0.3487             & 0.0645               \\
        61            & 0.0152          & 0.0000               & 0.2941             & 0.0349               \\
        66            & 0.0152          & 0.0000               & 0.2324             & 0.0069               \\
        67            & 0.0152          & 0.0000               & 0.3119             & 0.0556               \\
        \bottomrule
    \end{tabular}
    \caption{Peripheral nodes (Degree $\le$ 0.0152)} \label{tab:peripheral}
\end{table}

\subsubsection*{Hierarchy and Vulnerability}
These measures test the power structure and resilience of the network.

\paragraph{K-Core Decomposition}
The most densely connected "core" of the network is identified.  The analysis reveals a main core with a \textbf{k-value of 3} that consists of \textbf{20 nodes} (out of 63). The nodes in this core are: \texttt{[4, 8, 11, 12, 13, 15, 18, 19, 21, 24, 31, 32, 39, 41, 44, 47, 58, 59, 63, 64]}.
\begin{figure} [H]
    \centering
    \includegraphics[width=1\linewidth]{images/italian_network_core.png}
    \caption{Core of Italian Network graph visualization.}
    \label{fig:italian_core_net}
\end{figure}

\paragraph{Vulnerability Simulation}
Removing the three identified leaders (19, 63, 18) had a clear impact on the network structure. After their removal, the network broke into 18 components, and the largest connected part dropped from 65 to 24 nodes, showing a strong loss of cohesion. The density decreased from 0.0516 to \textbf{0.0288}. The average shortest path length increased from 3.0120 to \textbf{3.1341} (a \textbf{4.05\%} rise), indicating slightly longer communication chains among the surviving nodes.

\begin{figure} [H]
    \centering
    \includegraphics[width=1\linewidth]{images/italian_net_vuln.png}
    \caption{Italian Network sub-graph visualization without leaders.}
    \label{fig:italian_net_vuln}
\end{figure}


In summary, the \textbf{Italian gang network} is sparse but moderately cohesive, with clear community divisions and a defined inner core (\textbf{k = 3}). Leadership is concentrated in a few nodes (19, 63, 18), supported by brokers who link subgroups. When these leaders are removed, the network becomes \textbf{fragmented}, revealing limited redundancy and a \textbf{moderate structural vulnerability} despite its local cohesion.

\subsubsection{Ethnicity and Community Analysis}

The Italian gang network shows a \textbf{moderate level of ethnic homophily}, with an assortativity coefficient of 0.150. This indicates a slight preference for forming ties with individuals sharing the same \textit{Birthplace}, while the network as a whole remains well integrated.

The \textbf{mixing matrix} supports this interpretation: diagonal values are relatively high for groups 3, 4, and 5, reflecting intra-group cohesion, but many off-diagonal values remain substantial, revealing frequent \textbf{cross-ethnic interactions}.
\begin{table}[H]
    \centering
    \begin{tabular}{|c|c|c|c|c|c|c|c|c|c|}
        \hline
          & 1 & 2 & 3 & 4 & 5 & 6 & 7 & 8 & 9 \\ \hline
        1 & 0.018 & 0.000 & 0.031 & 0.004 & 0.004 & 0.000 & 0.022 & 0.009 & 0.000 \\
        2 & 0.000 & 0.000 & 0.000 & 0.000 & 0.000 & 0.000 & 0.000 & 0.000 & 0.004 \\
        3 & 0.031 & 0.000 & 0.096 & 0.053 & 0.031 & 0.000 & 0.061 & 0.004 & 0.004 \\
        4 & 0.004 & 0.000 & 0.053 & 0.228 & 0.013 & 0.009 & 0.035 & 0.000 & 0.004 \\
        5 & 0.004 & 0.000 & 0.031 & 0.013 & 0.009 & 0.000 & 0.026 & 0.000 & 0.009 \\
        6 & 0.000 & 0.000 & 0.000 & 0.009 & 0.000 & 0.000 & 0.000 & 0.000 & 0.000 \\
        7 & 0.022 & 0.000 & 0.061 & 0.035 & 0.026 & 0.000 & 0.000 & 0.000 & 0.000 \\
        8 & 0.009 & 0.000 & 0.004 & 0.000 & 0.000 & 0.000 & 0.000 & 0.000 & 0.000 \\
        9 & 0.000 & 0.004 & 0.004 & 0.004 & 0.009 & 0.000 & 0.000 & 0.000 & 0.000 \\
        \hline
    \end{tabular}
    \caption{Mixing matrix (proportion of connections between Birthplace groups) – Italian gang}
    \label{tab:mixing_matrix_italy}
\end{table}

Analysis of \textbf{centrality} shows clear differences among groups. Group 5 emerges as the most central, with the highest degree (0.159), betweenness (0.144), and eigenvector value (0.233). Groups 3 and 9 also show moderate centrality, suggesting intermediate structural roles, while groups 2, 6, and 7 appear more peripheral.
\begin{figure}[H]
    \centering
    \includegraphics[width=1\linewidth]{images/italian_centrality_by_birthplace.png}
    \caption{Boxplot of Italian centrality metrics by Birthplace}
\end{figure}

The \textbf{community analysis} identifies five communities with varying levels of diversity. Most are ethnically mixed, with the exception of community 4, dominated by groups 1 and 3. The mean Shannon index ($H = 1.174$) indicates \textbf{high community heterogeneity}.

\medskip

A significant proportion of ties (64.91\%) connect individuals from different \textit{Birthplace} categories, confirming a high degree of \textbf{cross-ethnic integration}. Thus, ethnic background does not represent a major organizational force within the network.

Subgraph analysis highlights additional structural differences:
\begin{itemize}
    \item Group \textbf{4} forms the largest and most cohesive internal cluster (21 nodes, density = 0.124).
    \item Group \textbf{5}, although consisting of only 2 members, is fully connected (density = 1.000).
    \item Groups \textbf{6–9} show little to no internal connectivity, relying mainly on cross-group ties.
\end{itemize}

\begin{table}[H]
    \centering
    \begin{tabular}{r r r r r}
        \toprule
        \textbf{Birthplace} & \textbf{Nodes} & \textbf{Edges} & \textbf{Density} & \textbf{Clustering} \\
        \midrule
        1 & 5  & 2  & 0.200 & 0.000 \\
        2 & 1  & 0  & 0.000 & 0.000 \\
        3 & 15 & 11 & 0.105 & 0.156 \\
        4 & 21 & 26 & 0.124 & 0.294 \\
        5 & 2  & 1  & 1.000 & 0.000 \\
        6 & 1  & 0  & 0.000 & 0.000 \\
        7 & 20 & 0  & 0.000 & 0.000 \\
        8 & 1  & 0  & 0.000 & 0.000 \\
        9 & 1  & 0  & 0.000 & 0.000 \\
        \bottomrule
    \end{tabular}
    \caption{Subgraph-level statistics by Birthplace – Italian gang}
    \label{tab:subgraph_italy}
\end{table}

\begin{figure}[H]
    \centering
    \includegraphics[width=1\linewidth]{images/italian_subgraphs.png}
    \caption{Italian subgraphs by Birthplace (ordered from 1 to 9)}
\end{figure}

In summary, the Italian gang shows \textbf{moderate homophily} but remains \textbf{highly integrated}. Most collaboration occurs across different national backgrounds, and ethnicity plays only a \textbf{secondary role} in shaping the network’s structure.
