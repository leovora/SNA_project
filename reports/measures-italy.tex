
\subsection{Italian gang}

\subsubsection{Structural analysis}

Using the Library Networkx \cite{networkx} from Python we studied different aspects of the italian gang network.

In figure \ref{fig:italian_net} a plot of the network is shown where each node is colored according to its country label.

\begin{figure} [H]
    \centering
    \includegraphics[width=0.7\linewidth]{images/italian_network.png}
    \caption{Italian Network graph visualization. Each node is colored according to its country label}
    \label{fig:italian_net}
\end{figure}

\subsubsection{Ethnicity analysis}

The analysis of the Italian gang network shows a \textbf{moderate tendency toward ethnic homophily}, with an assortativity coefficient of 0.150. This value suggests that individuals show a slight preference for connecting with others sharing the same \textit{Birthplace}, though overall the network remains relatively integrated. \\

The \textbf{mixing matrix} confirms this trend: while several diagonal values 
(e.g., for groups 3, 4, and 5) are higher (indicating intra-group cohesion) many off-diagonal entries are also non-negligible. This demonstrates a considerable number of \textbf{cross-national connections}, supporting the idea of partial ethnic mixing.

\begin{table}[H]
\centering
\caption{Mean centrality by Birthplace – Italian gang}
\label{tab:mixing_matrix_italy}
\begin{tabular}{|c|c|c|c|c|c|c|c|c|c|}
\hline
 & 1 & 2 & 3 & 4 & 5 & 6 & 7 & 8 & 9 \\ \hline
1 & 0.018 & 0.000 & 0.031 & 0.004 & 0.004 & 0.000 & 0.022 & 0.009 & 0.000 \\ 
2 & 0.000 & 0.000 & 0.000 & 0.000 & 0.000 & 0.000 & 0.000 & 0.000 & 0.004 \\ 
3 & 0.031 & 0.000 & 0.096 & 0.053 & 0.031 & 0.000 & 0.061 & 0.004 & 0.004 \\ 
4 & 0.004 & 0.000 & 0.053 & 0.228 & 0.013 & 0.009 & 0.035 & 0.000 & 0.004 \\ 
5 & 0.004 & 0.000 & 0.031 & 0.013 & 0.009 & 0.000 & 0.026 & 0.000 & 0.009 \\ 
6 & 0.000 & 0.000 & 0.000 & 0.009 & 0.000 & 0.000 & 0.000 & 0.000 & 0.000 \\ 
7 & 0.022 & 0.000 & 0.061 & 0.035 & 0.026 & 0.000 & 0.000 & 0.000 & 0.000 \\ 
8 & 0.009 & 0.000 & 0.004 & 0.000 & 0.000 & 0.000 & 0.000 & 0.000 & 0.000 \\ 
9 & 0.000 & 0.004 & 0.004 & 0.004 & 0.009 & 0.000 & 0.000 & 0.000 & 0.000 \\ 
\hline
\end{tabular}
\end{table}

When analyzing \textbf{centrality measures}, group 5 clearly stands out as the most central and structurally influential, with the highest mean degree (0.159), betweenness (0.144), and eigenvector centrality (0.233). Groups 3 and 9 also exhibit moderate centrality levels, suggesting participation in brokerage or connective roles. In contrast, groups such as 2, 6, and 7 show minimal centrality values, occupying peripheral positions within the network. 
Overall, influence appears somewhat concentrated but not monopolized by a single nationality. \\

\begin{figure}[H]
    \centering
    \includegraphics[width=1\linewidth]{images/italian_centrality_by_birthplace.png}
    \label{fig:placeholder}
\end{figure}

The \textbf{community analysis} identified five major communities. 
Most of these display mixed ethnic compositions, with only one cluster (community 4) being entirely dominated by two groups (1 and 3). The mean Shannon diversity index ($H = 1.174$) indicates high internal heterogeneity, suggesting that communities are composed of members from multiple national origins rather than segregated along ethnic lines.\\

Furthermore, 64.91\% of all connections occur between individuals of different 
Birthplace categories, a clear indicator of \textbf{strong cross-ethnic integration}. 
This finding supports the interpretation that ethnic background is not a dominant organizing factor in the structure of the Italian gang. \\

Subgraph analysis by \textit{Birthplace} reveals additional information:
\begin{itemize}
    \item Group \textbf{4} (21 nodes, density = 0.124) forms the largest and most internally cohesive subcommunity.
    \item Group \textbf{5}, although small (2 nodes), is fully interconnected (density = 1.000), 
    representing a tightly bonded dyad.
    \item Other groups (6, 7, 8, 9) show limited or no internal links, implying 
    \textbf{dependence on inter-ethnic ties} for maintaining connectivity.
\end{itemize}

\begin{table}[H]
\centering
\caption{Subgraph-level statistics by Birthplace – Italian gang}
\label{tab:subgraph_italy}
\begin{tabular}{|c|c|c|c|c|}
\hline
\textbf{Country} & \textbf{Nodes} & \textbf{Edges} & \textbf{Density} & \textbf{Clustering} \\ \hline
1 & 5  & 2  & 0.200 & 0.000 \\ 
2 & 1  & 0  & 0.000 & 0.000 \\ 
3 & 15 & 11 & 0.105 & 0.156 \\ 
4 & 21 & 26 & 0.124 & 0.294 \\ 
5 & 2  & 1  & 1.000 & 0.000 \\ 
6 & 1  & 0  & 0.000 & 0.000 \\ 
7 & 20 & 0  & 0.000 & 0.000 \\ 
8 & 1  & 0  & 0.000 & 0.000 \\ 
9 & 1  & 0  & 0.000 & 0.000 \\ 
\hline
\end{tabular}
\end{table}

\begin{figure}[H]
    \centering
    \includegraphics[width=1\linewidth]{images/italian_subgraphs.png}
    \label{fig:placeholder}
\end{figure}

In summary, the Italian gang exhibits \textbf{moderate homophily but high overall integration}, 
with collaboration patterns largely transcending national divisions. 
Ethnicity appears to play a \textbf{secondary role} in shaping relational dynamics.