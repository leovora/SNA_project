\section{Critique}
\label{critique}
This study effectively provided meaningful insights into both key objectives, successfully outlining the core topological and structural features of both networks, and demonstrating that ethnicity is not a unique, monolithic driver for ties between members in the analyzed network. \\

However, these findings are inevitably constrained by the quality of the data itself. As discussed in the Validity section, the datasets represent simplified, static an somehow incomplete models of the real organizations, so the reconstructed topologies are necessarily approximations, especially considering that covert networks are only partially observable. Specifically, the choice to treat both networks as unweighted allowed for an easier comparison, but limited the fidelity of the analysis, since it flattened the distinction between weak interactions and strong ties. In addition, the two datasets differ in the richness and granularity of nodes metadata: the Italian one only provides a country-of-origin variable, while the London dataset contains numerous social and behavioral metadata, which again have been neglected to maintain comparability between networks, in spite of fidelity.\\

Another limitation arises from the static nature of the model: criminal networks evolve over time, yet the analysis was carried out on a single snapshot for each of the organizations. A temporal analysis could have offered a deeper understanding of the investigated phenomena, if the data had been available.\\

In addition, the data limitations observed here also point to a broader issue that extends beyond the data quality of the specific instances that have been examined: ethnic homophily in covert criminal networks is a complex phenomenon, whose analysis cannot be fully addressed through isolated network instances. Thoroughly addressing this goal requires a broader effort on a much larger and more detailed empirical basis, which incorporates more complex and attribute-richer datasets.\\

Despite the discussed limitations, the analysis still offers meaningful insights into the structural characteristics and dynamics of the examined networks, contributing to a clearer understanding on the organizational and interaction patterns of these groups, and providing a useful foundation to future research.