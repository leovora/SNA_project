\subsection{Comparison}
This section presents a comparative analysis of the two covert criminal networks, in which we examine their structural metrics in order to understand the distinct organizational structures of each group.

\subsubsection{Topology and Cohesion}

The structural properties of the two networks reveal markedly different organizational patterns: the Italian Network is considerably sparser than the London one, a difference evident in both its lower \textbf{density} (Italian: 0.0516, London: 0.2201) and its smaller \textbf{average degree} (Italian: 3.0430, London: 11.67). This suggests that interpersonal ties in the Italian gang are more selectively-distributed, and that interactions are far less intense, while the London gang functions as a highly integrated, tightly-knit community.

Consistently, the Italian network exhibits a longer \textbf{average path length} (Italian: 3.012, London: 2.05) and a larger \textbf{diameter} (Italian: 6, London: 4), indicating a lower network efficiency. On the other hand, the London Gang constitutes a more structurally efficient environment in which information and communication can propagate more quickly, requiring, on average, only two intermediaries.

Local cohesion reinforces this distinction: the \textbf{clustering coefficient} is higher in the the London network (London: 0.6331, Italian: 0.4347), indicating a network that is richer in transitive relationships. The most telling difference in the topology of the networks lies in the community structure. The Italian network displays higher \textbf{modularity} (Italian: 0.5561, London: 0.2665), which suggests it is organized into distinct sub-groups or cells. These modules are typically internally cohesive but are only sparsely connected to each other. On the other hand, the the London gang’s low modularity, combined with its high density, indicates a lack of significant sub-divisions. It operates as a single, large, cohesive core group rather than a confederation of smaller cells.

\subsubsection{Centrality Measures: Hierarchy, Leadership and Brokerage}

The analysis of centrality measures of the two networks highlights a clear hierarchical difference. In particular, a clear and steep hierarchy can be observed in the Italian network. The three identified leaders (nodes 19, 63, 18) possess high scores across all centrality measures, while the identified peripheral members are approximately 30 times less central. This aligns with the network’s high modularity: these leaders function as \textbf{brokers}, bridging the structural holes between the otherwise separated cells.

In the London Gang, the hierarchy appears flatter, as it shows a more balanced distribution of centrality. In particular, two core actors (nodes 1 and 7) occupy both leadership and brokerage positions, but the overall gap between such central nodes and the peripheral ones is smaller—about one order of magnitude.

Furthermore, the identified leaders have high \textbf{degree} and \textbf{closeness}, but relatively low \textbf{betweenness}, which shows they act as “hubs” in the center of a single, dense core, rather than proper brokers connecting disparate groups.

\subsubsection{K-Core and Robustness}

The K-Core analysis provides an additional insight into the robustness of these two organizations. The Italian network culminates in a modest \textbf{3-core} composed of 20 nodes,  reinforcing its sparse nature.

As for robustness, removing the identified leaders caused a noticeable increase in average path length and a slight reduction in density, indicating the central role of these individuals in maintaining short communication distances within the core. 

On the contrary, the London network exhibits a much more cohesive, robust structure, featuring a \textbf{11-core} composed of 13 nodes, which suggests a group with strong internal redundancy and resilience: even after removing key leaders, the network density remains fairily large and the increase in average path length is modest.

\subsubsection{Ethnicity Analysis}

This section examines how members’ ethnicity influences the relational structure of each group, with the aim of discovering to which extent sociocultural factors contribute to the internal functioning of each organization.

While both networks operate in distinct geographical contexts, they share a foundational characteristic: a high degree of integration across ethnic lies. More specifically, the tendency towards ethnic homophily, as measured by the assortativity coefficient, is positive, but remains weak in both cases. The Italian network exhibits a slightly higher coefficient (0.150) compared to the London network (0.113), yet the proximity and the magnitude of the two values indicate that while members may have a slight preference for forming ties with individuals of the same origin, ethnicity harmony is not a dominant structural driver in either case. 

This observation is strongly supported by the proportion of cross-ethnic connections. The two networks are remarkably similar in this regard, with 64.91\% of ties in the Italian gang and 64.44\% of ties in the London gang occurring between individuals of different ethnicity groups.

This surprising similarity highlights a common underlying dynamic: both organizations are fundamentally integrated and heavily rely on cross-national cooperation, rather then acting as ethnically segmented groups. The mixing matrices for both networks corroborate these observations: each matrix contains high diagonal values for specific ethnic groups, yet both exhibit non-negligible cross-diagonal values.

Despite this high-level similarity, the two networks exhibit a key difference in their internal community structure, highlighted by the community analysis. In particular, the Italian network, which features a higher Shannon Diversity Index of H = 1.174, demonstrates an overall higher level of internal heterogeneity, meaning that most communities integrate members from multiple birthplaces, with only one (Community 4) being clearly dominated by two ethnic groups.

Conversely, the London Network exhibits lower community diversity, with a Shannon diversity index of H = 0.886, showing that several communities (mainly communities 1, 2 and 3), show strong ethnic concentrations. This pattern indicates that, while cross-ethnic ties remain common at the macroscopic level, the members of this network tend to cluster into ethnically homogeneous groups at the community level.

Concerning the distribution of influence of the different ethnic groups, it can be observed that the Italian network exhibits a relatively concentrated influence structure, with ethnicity group 5 clearly emerging as the most central and influential. In contrast, the London network exhibits a more distributed influence structure, where centrality is not monopolized by a single ethnic group. Instead, it is shared primarily by ethnic groups 1 and 4, with group 2 also playing an important role.
